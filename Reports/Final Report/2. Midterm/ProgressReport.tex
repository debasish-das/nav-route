\documentclass{article}
\usepackage{graphicx} % Required for inserting images

\title{Midterm Progress Report}
\author{
  Debasish Das\\
  \texttt{d.das@student.fdu.edu}
  \and
  Sai Nithish Reddy Madireddy\\
  \texttt{s.madireddy@student.fdu.edu}
  \and
  Sanad Mukadam\\
  \texttt{s.mukadam@student.fdu.edu}
}
\date{23 June 2023}

\bibliographystyle{plain}
\usepackage[hyphens,spaces]{url}

\begin{document}

\maketitle

\section{Project Overview}
The primary objective of our project, named "nav-route," is to devise a web application offering optimized routing solutions based on a host of parameters such as distance, safety, time, and cost. The primary beneficiaries of such a solution would be travelers, delivery persons, business people, and basically anyone needing to visit multiple destinations from a specific source. The project seeks to fill a gap in the current navigation applications by allowing users to prioritize their unique preferences and receive a sequence of destinations that will either minimize the total distance or time depending on the selected priorities.

\section{Accomplishments}
As of the mid-semester checkpoint, our team has managed to successfully complete MVP. Here are some of the specific milestones we have achieved listed under each team member responsible for driving them:

\subsection{Application: Debasish Das}
Worked on a simple web page what uses JavaScript library for ‘Google Maps Platform’ and ‘Bootstrap v5.0.2’ CSS. It allows a user to select a starting point and multiple destinations. Selecting a location comes with autocomplete feature using ‘Google Maps API’. Then the user can get all the routes possible to traverse the destinations. ‘Heap Algorithm’ is used to find the permutation of the destinations \cite{r1}. Then each result of permutation was mapped on the distance matrix generated from ‘Google Maps API’. If the number of destinations is ‘n’, the time complexity is O(n!). n! number of routes would be generated. I also tried Dijkstra and Floyd-Warshall’s algorithms, but found Heap algorithm more suitable to work on the distance matrix from ‘Google Maps API’. The generated routes are shown in the web page with detailed information (i.e., sequence of places, time and destinations). These routes can be filtered based on user’s choice link shortest distance, priority and priority. By default, the routes are showed as ascending order of total distance covered. Selecting a route will show it on Google Map.\\
\\
For simplicity and the technology stack that is familiar to all, I used Modular ECMAScript of JavaScript. However, it would be better to use React framework with NodeJS for an efficient development environment.


\subsection{Openstreet Map API: Sai Nithish Reddy Madireddy}
During the initial week, the focus was directed towards the creation of a sample file capable of accepting latitude and longitude inputs and subsequently displaying the corresponding map. Furthermore, exploration of alternative solutions, available through Google Maps, was undertaken during this phase. In the second week, further progress was made by the integration of authentic coordinates into the sample file. A collection of 20-25 actual coordinates was assembled and subsequently incorporated as inputs within the file. Consequently, when the source and destination were selected, the map duly presented the intended points.


\subsection{UI Design:Sanad Mukadam}
The Address lookup and autocomplete feature is an important aspect of our project. It is important for the user to be able to type in an address and be able to access autocompletion of addresses based on suggestions generated by the API used for lookup.
Technologies Used: 
•	Web – HTML 5, CSS3, Bootstrap 5, JavaScript
•	IDE: Visual Studio Code 1.79
•	Graphic Editor: Adobe Photoshop

A webpage was developed utilizing HTML, CSS, JavaScript, and Bootstrap, enabling the user to enter an address while benefiting from the type-ahead-search functionality of the Google map search field. The autocomplete service can match on full words and substrings, resolving place names and addresses.
Google Maps API was embedded to fulfill the task. The developed webpage provides the user with two options:
•	To input an address in a single text box,
•	To input an address in the initial text box, which will then autofill other details such as street address, city, state, zip code, and country.
The completed webpage was subsequently uploaded to the GitHub repository.
Screenshot:
 
UI Design Hand-sketched Concept:
The UI is an important part of any application. Several sketches were prepared after consultation with the team and agreed upon drafts were uploaded to the Git repository. At the end, we intend to integrate the application with the front end. The concept drawing of the UI, therefore was a great idea that allowed us to envision how we wanted the app to look and function. Also, it offers a unique way to anticipate the flow of the program and allows us to take steps in making the interface easy to use and maintain.
Following are some of the screens we designed:





Main Screen – Login, Registration
 
Registration
 




Destination Waypoints Selection
 
Routes Screen
 




Navigation
 
UI Design Implementation using HTML/CSS and JavaScript:
Technologies Used: 
•	Web: HTML 5, CSS3, Bootstrap 5, JavaScript
•	IDE: Visual Studio Code 1.79

•	Webpages were developed utilizing HTML, CSS3, Bootstrap 5, and JavaScript. Google API was embedded for map functionality, enabling the display of route maps. 
•	A nav-route module was developed, including a welcome page with an engaging animated background. This feature allows users to log in, register, enter a destination, and add additional destinations.
•	Based on the input for destinations, a route plan webpage was created detailing stops 1, 2, 3, 4, where users have the option to add or remove stops. The total distance and total stops are displayed depending on the route selection. Users have the option to reset the input fields or continue with their current inputs.
•	Users can view the route map for the selected stops based on their chosen route plan.
•	A user interface was developed offering two options:
o	Light-colored-animated-background-theme
o	Dark-colored-background-theme



Screenshots: 
Two versions of UI themes were explored - a light and a dark version, and below are some of the pages developed. The darker version of the UI was also explored as an option.

\section{Conclusion}
As we approach the mid-point of our project timeline, we are pleased to report that we are on track, having achieved half of our proposed objectives. We are confident that our comprehensive planning, dedicated teamwork, and problem-solving approaches will carry us through the rest of the project. Our ultimate goal is to deliver a functional and user-friendly web application that can provide optimized routes based on user preferences, enhancing travel efficiency for a wide variety of users. We look forward to the challenges and learning opportunities that lie ahead as we work towards the completion of our project.

\bibliography{ProgressReport.bib}

\end{document}
